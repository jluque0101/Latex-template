\input{preambulo.tex}
\usepackage{url}
\usepackage[linktocpage]{hyperref}
\usepackage{xcolor}
\usepackage{listings}
\lstset{basicstyle=\ttfamily,
	showstringspaces=false,
	commentstyle=\color{blue},
	keywordstyle=\color{red}
}
%----------------------------------------------------------------------------------------
%	TÍTULO Y DATOS DEL ALUMNO
%----------------------------------------------------------------------------------------

\title{	
\normalfont \normalsize 
\textsc{{\bf Asignatura (2014-2015) Grupo A} \\ Grado en Ingeniería Informática \\ Universidad de Granada} \\ [25pt] % Your university, school and/or department name(s)
\horrule{0.5pt} \\[0.4cm] % Thin top horizontal rule
\huge Título (Nombre y número) \\ % The assignment title
\horrule{2pt} \\[0.5cm] % Thick bottom horizontal rule
}
\author{José Manuel Luque Burgos - 15452527T} % Nombre y apellidos

\date{\normalsize\today} % Incluye la fecha actual




%----------------------------------------------------------------------------------------
% DOCUMENTO
%----------------------------------------------------------------------------------------

\begin{document}

\maketitle % Muestra el Título

\begin{figure}[H] %con el [H] le obligamos a situar aquí la figura
	\centering
	\includegraphics[scale=0.8]{ugr.jpg}  %el parámetro scale permite agrandar o achicar la imagen. En el nombre de archivo puede especificar directorios
	\label{ugr}

\end{figure}

\newpage %inserta un salto de página %clearpage


\tableofcontents % para generar el índice de contenidos

\listoffigures

\newpage

%----------------------------------------------------------------------------------------
%	SECCIONES
%----------------------------------------------------------------------------------------

\section{Seccion}
\subsection{Subseccion}
Parrafo
\newline
Cita
\cite{cita1b}
\newline
lista:
\begin{itemize}
	\item elemento
	\item sublista enumerada

	\begin{enumerate}
		\item elemento
		\item elemento
	\end{enumerate}
\end{itemize}

asdf
\newline

\begin{figure}[H] %con el [H] le obligamos a situar aquí la figura
	\centering
	\includegraphics[scale=1]{a.JPG} 
	\caption{Configuración J2, J3, J4 y J5}
	\label{fig:a}
	
\end{figure}

Referencia a figura \ref{fig:a} 
%------------------------------------------------


\newpage

%----------------------------------------------------------------------------------------
%	BIBLIOGRAFÍA
%----------------------------------------------------------------------------------------
\bibliographystyle{plain} % hay varias formas de citar

\bibliography{citas} %archivo citas.bib que contiene las entradas 

\end{document}